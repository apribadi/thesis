\documentclass[11pt]{article}
\usepackage{pset}

\newcommand*{\X}{\mathfrak{X}}
\newcommand*{\Mod}{\mathcal{M}}

\begin{document}

\noindent A really simple nontrivial model -- naive Bayes.

Before we think about classification, we should just think about probability
distributions.  Suppose that we have a random variable $X = (X_1, X_2)$ that
takes values in $\{0,1\}^2$.  There are $4$ possible values, so the distribution
of $X$ is a point in the 3-dimensional probability simplex
\[
    \Delta_3 = \left\{(p_{00}, p_{01}, p_{10}, p_{11}) \mid \sum p_{ij} = 1, p_{ij} \ge
    0\right\} \subset \R^4
\]
where the coordinate $p_{ij}$ is the probability that $(i, j)$ occurs.

If we make the assumption that the components $X_1$ and $X_2$ are independent,
we introduce the constraints
\[
    p_{ij} = (p_{i+})(p_{+j}) = (p_{i0} + p_{i1})(p_{0j} + p_{1j})
\]
where a $+$ in an index position indicates summation over that index, e.g.
$p_{i+} = p_{i0} + p_{i1}$.  These are quadratic polynomial equations.  Then our
model is
\[
    \mathcal{M}
        = \left\{(p_{00}, p_{01}, p_{10}, p_{11}) \mid \sum p_{ij} = 1, p_{ij}
        \ge 0, p_{ij} = (p_{i+})(p_{+j})
    \right\} \subset \Delta_3 \subset \R^4
\]
which is a semialgebraic set.

We can introduce an explicit parametrization of $\mathcal{M}$.  Specifically
\begin{IEEEeqnarray*}{rCrCl}
    f &:& \{(u, v) \mid 0 \le u,v \le 1\} &\to& \mathcal{M}\\
     && (u,v)      &\mapsto& (uv, u(1-v), (1-u)v, (1-u)(1-v))
\end{IEEEeqnarray*}
is a polynomial parametrization from a 2-dimensional space (which I think would
be considered an isomorphism of semialgebraic sets?).  Here $u = p_{0+}$ and $v
= p_{+0}$.

In order to examine a binary classification problem, we would consider the joint
distribution over $\{0,1\}^2 \times \{0, 1\}$ and introduce appropriate
independence conditions there.  With coordinates $p_{ijk}$, I think the
conditions are
\[
    p_{ijk} = (p_{i+k})(p_{+jk})
\]
but I would need to check.  Then given an observation $(i, j)$, the probability
of the label $0$ would be $p_{ij0} / p_{ij+}$ and the probability of the label
$1$ would be $p_{ij1} / p_{ij+}$.


\end{document}
