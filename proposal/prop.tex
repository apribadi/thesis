\documentclass[11pt]{article}
\usepackage{pset}

\newcommand*{\X}{\mathfrak{X}}

\begin{document}

\section{Intro}

builds off of

Toric ideals and graphical models

geometry of the restricted boltzmann machine

geometry of bayesian networks

tropical geometry of statistical models

\linespace
Watanabe algebraic geometry

IGO

\linespace
sum-product networks

\linespace
\noindent prior art in deep learning

efficient algorithm for deep belief networks (contrastive divergence)

deep boltzmann (?)


\section{Graphical models}

A graphical model is a probabilistic model for which a graph (directed or
undirected) represents the conditional independence structure between random
variables. 

Let $G$ be a graph with vertex set $V$.  Suppose that $(X_\alpha)_{\alpha \in
V}$ are random variables taking values in probability spaces
$(\X_\alpha)_{\alpha \in V}$.  Usually the $X_\alpha$ are either discrete or
real-valued.  The joint probability of $X = (X_\alpha)$ is said to
\emph{factorize} with respect to $G$ if
\[
    P[X_1, \ldots, X_n] = 
        \prod_\alpha P[X_\alpha \mid X_\beta, \beta \text{ is a parent of }
        \alpha]
\]
that is, random variables have non-trivial dependencies only if they are
neighbors.  We can also formulate an equivalent statement in terms of
conditional independence statements.

In the case of an undirected graph, via the Hammersley–Clifford theorem it is
known that the probability density factors as
\[
    P[X_1, \ldots, X_n] = 
        \prod_{S \le G} P[X_\alpha \mid X_\beta, \beta \in S]
\]
where the $S$ are complete subgraphs (cliques) of $G$.

Graphical models turn up in a number of places -- Bayesian networks (directed)
and Markov random fields (undirected) in statistics, Boltzmann machines in
machine learning, and Ising models (spins of particles in a lattice) in
statistical mechanics.

\section{Example: Boltzmann machines}

A Boltzmann machine factors with respect to an undirected graph, and in addition
assumes that the random variables only have pairwise interactions.

Concretely, let the random variable $X = (X_i)_{1 \le i \le n}$) take values in
$\{0,1\}^n$, and let $\theta = (\theta_{ij})$ be a symmetric $n \times n$ matrix.
Then $\theta$ parametrizes a family of probability distributions.

We can regard 
\[
    H_\theta(X) = \sum_{i < j} \theta_{ij} X_i X_j + \sum_i \theta_i X_i
\]
as the energy of the system.  Then the probability of $X$ is
\[
    P_\theta[X] = \frac{\exp H_\theta(X)}{Z_\theta}
\]
where $Z_\theta$ is the normalizing factor
\[
    Z_\theta = \sum_{X \in \{0,1\}^n} H_\theta(X)
\]
known as the partition function.

This probability model is, of course, inspired by statistical mechanics where it
is known as the Ising model.  Boltzmann machines have recently become important
in certain types of machine learning.


\section{Geometry of Families of Probability Distributions}

A probability distribution over a finite set corresponds with a point in the
simplex 
\[
    S_n = 
    \left\{(x_1, \ldots, x_{n}) \mid x_i \ge 0, \sum_{i=1}^n x_i= 1\right\} 
    \cong 
    \left\{(x_1, \ldots, x_n) \mid x_i \ge 0, \sum_{i=1}^{n-1} \le 1\right\}
\]
of appropriate dimension.

Then the possible

information geometry (Riemannian geometry)

algebraic geometry

tropical geometry

optimization - maximum a priori, BIC, AIC

algorithms, MCMC


\section{A primer on machine learning}
\subsection{Goals}
foo bar
\subsection{Techniques}
baz quz
\subsection{Deep learning}
asdf 

\end{document}
