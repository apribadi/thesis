\documentclass[11pt]{article}
\usepackage{pset}

\newcommand*{\X}{\mathfrak{X}}
\newcommand*{\Mod}{\mathcal{M}}

\begin{document}

\section{September 9, 2011}

\subsection{Introduction}

I would like to work on a problem in algebraic statistics, and more specifically
a problem about graphical models.  I do not yet have a coherent `story' for the
background of a problem, so I will instead discuss some topics in a close
neighborhood of what I want to work on.


\subsection{Graphical models}

Independence conditions arise naturally in many probabilistic models.  A
graphical model is a probabilistic model for which a graph (directed or
undirected) represents the conditional independence structure between random
variables. 

Let $G$ be a graph with vertex set $V$.  Suppose that $(X_\alpha)_{\alpha \in
V}$ are random variables taking values in probability spaces
$(\X_\alpha)_{\alpha \in V}$.  Usually the $X_\alpha$ are either discrete or
real-valued.  The joint probability of $X = (X_\alpha)$ is said to
factor with respect to $G$ if the probability density is
\[
    P[X_1, \ldots, X_n] = 
        \prod_\alpha P[X_\alpha \mid X_\beta, \beta \text{ is a parent of }
        \alpha]
\]
that is, the dependence of a random variable on the other variables is
completely captured by its dependence on its immediate parents (i.e. on the
variables having directed edges into it).  We can also formulate an equivalent
statement in terms of conditional independence statements.

In the case of an undirected graph, it is known (the Hammersley–Clifford
theorem) that the probability density factors as
\[
    P[X_1, \ldots, X_n] = 
        \prod_{S \in C(G)} P[X_\alpha \mid X_\beta, \beta \in S]
\]
where the $S$ are complete subgraphs (cliques) of $G$.

Graphical models turn up in a number of places -- some include Bayesian networks
(directed) and Markov random fields (undirected) in statistics, Boltzmann
machines in machine learning, and Ising models (particle spins in a lattice) in
statistical mechanics.

\subsection{Boltzmann machines}

A Boltzmann machine factors with respect to an undirected graph, and in addition
assumes that the variables only have pairwise interactions.

Concretely, let the random variable $X = (X_i)_{1 \le i \le n}$ take values in
$\{0,1\}^n$, and let $\theta = (\theta_{ij})$ be a symmetric $n \times n$
matrix.  Then $\theta$ parametrizes a family of probability distributions.

We can regard 
\[
    H_\theta(X) = \sum_{i < j} \theta_{ij} X_i X_j + \sum_i \theta_i X_i
\]
as the energy of the system.  Then the probability of $X$ is
\[
    P_\theta[X] = \frac{\exp(- H_\theta(X))}{Z_\theta}
\]
where $Z_\theta$ is the normalizing factor
\[
    Z_\theta = \sum_{X \in \{0,1\}^n} \exp(-H_\theta(X))))
\]
known as the partition function.

Notice that the `exponent of sum' form of the probability density directly
corresponds to a `product of terms' form, from which we can derive conditional
independence statements if some of the entries $\theta_{ij}$ are zero.

The Boltzmann machine is inspired by ideas statistical mechanics, where the
distribution is known as the Ising model.  Also, restricted Boltzmann machines
(RBMs) have recently become a building block for an important technique in
machine learning -- deep belief networks.

A RBM has some number of `observed' variables $X_1, \ldots, X_m$ and hidden
variables $X_{m+1}, \ldots, X_n$.  The graph for the RBM is the complete
bipartite graph, where every edge is between an observed and hidden variable,
i.e. $\theta_{ij} = 0$ if $i,j \le m$ or $i,j \ge m+1$.

Because the hidden variables are not observed, in order to fit the model's
parameters to the data we consider the marginalized probabilities
\[
    P_\theta[X_1, \ldots, X_m] = \sum_{(X_{m+1}, \ldots, X_n) \in \{0,1\}^{n-m}}
        P_\theta[X_1, \ldots, X_n].
\]
With a fitted parameter $\theta$ and an observed value $X_1, \ldots, X_m$, we can
also compute a probability distribution over the hidden variables; in some
sense, the hidden variables constitute an `explanation' for the data.


\subsection{The (algebraic) geometry of probability models}

A probability distribution over a finite set of size $n$ corresponds with a
point in the \emph{probability simplex}
\[
    \Delta_{n-1} = 
    \left\{(x_1, \ldots, x_{n}) \mid \sum_{i=1}^n x_i= 1, x_i \ge 0 \right\} 
    \cong 
    \left\{(x_1, \ldots, x_n) \mid \sum_{i=1}^{n-1} x_i \le 1, x_i \ge 0 \right\}
\]
of dimension $n-1$.  We can think of a probability model as a subspace 
$\Mod \subset \Delta_{n-1}$.

One way to select a particular probability distribution $p$ from a model $\Mod$
is by \emph{maximum likelihood}.  That is, given observations $X^{(1)}, \ldots,
X^{(N)}$ we attempt to maximize
\[
    L(p \mid X^{(1)}, \ldots, X^{(N)}) = \prod_{i=1}^N p(X^{(i)}
\]
though in practice the negative log-likelihood $l(p, X^{(i)}) = -\log L(p, X^{(i)}$ is
often used (this is a `loss' function, and the losses of the observations are
summed).

From the viewpoint of algebraic geometry, we will usually stipulate that models
are semialgebraic sets, that is, subsets of $\R^n$ that are determined by
polynomial equations and polynomial inequalities.  Real algebraic geometry
(which studies these objects) can be difficult, so sometimes the field is
extended to the complex numbers and/or the Zariski closure of the set is
considered.

Oftentimes, to compute things we will want some rational parametrization $\R^n
\supset U \to \Mod$ of the model.

Using the computational tools available to algebraic geometry (and with problem
sizes, e.g. dimension, that are extremely small) it is sometimes possible to
answer questions like `what is the maximum likelihood parameter' exactly.

The overall hope of approaches like this is (from what I can see) to apply the
tools of geometry to the problems of statistics.  (Compare with `information
geometry', the application of the techniques of differential (esp. Riemannian)
geometry to problems in probability theory, pioneered in the 1980s -- I don't
know how much it has caught on.)  One potential gain from this approach is that
`coordinate-invariance' comes more naturally than usual.


\subsection{Machine learning (and deep learning)}

Machine learning involves algorithmically taking data in order to perform better
at some task.  Abstractly, its goal is similar to that of statistics -- to learn
from data.  The emphasis, however, tends to be toward large data sets, efficient
algorithms (for those large data sets), and fewer assumptions on models
(`non-parametric' statistics).  

Machine learning is studied both by computer scientists and statisticians (who
sometimes call it statistical learning).  (A very good introduction, and one
from which I have taken much of my viewpoint on ML, is \textit{The Elements of
Statistical Learning} by Hastie, Tibshirani, and Friedman -- three
statisticians.)

One very common task is the classification problem.  Here, observations are
points in some (complicated, high-dimensional) space $X$, and the goal is to
assign an observation $x \in X$ to a class label from some finite set $\{A, B,
\ldots \}$.  It can be viewed as a problem where we are attempting to
approximate a function $f : X \to \{A, B, \ldots\}$ given a number of observed
class labels $f(x^{(i)})$.  Of course, this problem is impossible without some very
strong constraints on the type of function (the `model').  You can transform the
classification problem into a problem of estimating a probability distribution
by simply considering the joint distribution over observations and labels.

Most successful algorithms are in some sense `only a step or two away from
linear'.  A basic approach (to, say, binary classification) is to try to
separate the points in the two classes by some hyperplane.  A modification (used
by support vector machines (SVMs)) is to transform the data to some even-higher
dimensional space (infinite, perhaps), and to separate by a hyperplane there.
Another common approach is to have some family of basis functions, and to
consider linear combinations of them.

However, for some problems these techniques may not be enough.  This has
prompted (with some success only in the last ten years?) the creation of
so-called `deep learning' methods, which have `many layers of non-linearity'.
One very important method (perhaps the \emph{most important} so far) is called
the deep belief network and involves stacking restricted Boltzmann machines and
training them greedily.  (Professor Keller, I think, has done some work with
deep belief networks.)  

Deep belief networks can be, however, extremely difficult to optimize.
Monte-Carlo-Markov-Chain (MCMC) techniques are necessary, and esoteric sounding
things like `importance sampling' and `simulated annealing' are used to improve
performance.  (The general subject area is called `stochastic optimization', I
think.  Remember that we do not need exactly optimal solutions, but only
solutions that are `good enough'.)

These and similar methods have performed well on some problems.  One `standard'
example that shows up a lot in papers is the MNIST dataset of handwritten
digits.  In some cases, these algorithms solve an even harder problem than what
humans do -- the `permutation invariant' problem delivers the pixels of the
image as a vector with no a priori knowledge of the geometry of the image.

\subsection{Prior work}

For the algebraic geometry of graphical models, one paper is `On the toric
algebra of graphical models' (Geiger, Meek, Sturmfels -- 2006).  A book-length
account that covers graphical models, among other topics in algebraic
statistics, is `Lectures on Algebraic Statistics' (Drton, Sturmfels, Sullivant
-- 2009).  

Papers that have focused on specific types of graphical models include
`Algebraic Geometry of Bayesian Networks' (Garcia, Stillman, Sturmfels -- 2003)
and `Geometry of the restricted Boltzmann Machine' (Cueto, Morton, and Sturmfels
-- 2009).  

Inspired by questions in phylogenetics, tropical geometry has been applied to
questions in algebraic statistics (incl. graphical models).  See the paper
`Tropical Geometry of Statistical Models' (Pachter, Sturmfels -- 2003) and the
book `Algebraic Statistics for Computational Biology' (Pachter, Sturmfels --
2005).

Another book, which builds theory for considering optimization problems from the
viewpoint of algebraic statistics is `Algebraic Geometry and Statistical
Learning Theory' (Watanabe 2009).

From the machine learning side, questions in deep learning have spawned a large
number of papers, but an early and important one is `A fast learning algorithm for
deep belief nets' (Hinton 2007).  A good survey paper on the need for and
approaches to deep learning is `Learning deep architectures for AI' (Bengio
2009).

One very recent type of graphical model that encompasses models with tractable
partition functions was introduced in `Sum-Product Networks: A New Deep
Architecture' (Domingos, Poon 2011).

\subsection{Where to go from here}

One possibility for a problem to work on is to simply extend Sturmfels' work on
the restricted Boltzmann machine.  There is some distance yet between examining
its geometry and having something useful to say about the optimization problem.

\end{document}
